\chapter{Conclusions}

The goal of this thesis was to research feasibility of building distributed computing system based on volunteer computing. Such system could be used to create Internet computing markets and potentially reduce needs to employ commercial computing grids, while encouraging both commercial and academic institutions to use distributed computing - because of lowering the costs of using the method. Part of lowering the costs could be the ability to use currently owned hardware and connect it to the system as computing nodes, therefore getting some of the spent money back.

A prototype was created, to investigate technical difficulties of building such system. Prototype proved that use of BitTorrent is feasible way of distributing projects, and that VirtualBox can be used as virtualization technology in order to provide promised flexibility of defining computing projects. 

Simulations were performed to see dynamics of volunteer computing system and research algorithms needed to efficiently compute work in distributed way. Simulations have shown that proposed algorithm can be used to provide a way of collecting and verifying results in distributed computing system. However, defense from coordinated attacks using lots of colluding malicious nodes turned to outside of the scope of this thesis, because it is a huge challenge on its own. Special protections would have to be designed and employed just to protect from such attacks.

