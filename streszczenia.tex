\streszczenie{
    The goal of this thesis was to design a distributed computing system based on the idea of volunteer computing. Concept of computing market is presented, where owners of spare computing power can sell their resources to people who need to perform computation. Means of validating correctness of the result is proposed, which is based on a model of measuring trust of nodes.

    In order to test the trust model, a simulator was created. The simulator can simulate computing network of any number of nodes and any number of jobs. The thesis describes the simulator, as well as results of simulations performed. Conducted experiments have shown that presented model is valid, and can prevent some degree of malicious activity in the network.

    Prototype of volunteer computing system is presented. Prototype uses lightweight virtual machines as container for computing programs. Oracle VirtualBox technology is used. BitTorrent protocol is used to optimize virtual machine data transfer from server to the nodes. Prototype is implemented in Node.js.

    Thesis ends with conclusions. Problems with this solutions have been identified and noted. The chapter also shows possible way in how to improve on the idea of volunteer computing.
}

\streszczeniepl{
    Celem pracy dyplomowej było zaprojektowanie systemu do obliczeń rozproszonych przeprowadzanych na bazie wolontariatu. Zaprezentowana jest koncepcja rynków obliczeniowych, gdzie posiadacze nieużywanych zasobów komputerowych mogą sprzedawać zasoby osobom które potrzebują przeprowadzić obliczenia. Zaproponowano mechanizm sprawdzania poprawności wyników, który oparty jest na modelu określania zaufania poszczególnych węzłów biorących udział w obliczeniach.

    W celu przetestowania proponowanego modelu zaufania, stworzony został symulator, który symuluje sieć o dowolnej ilości węzłów oraz zadań. Praca zawiera opis symulatora, a także wyniki przeprowadzonych symulacji. Symulacje pozwoliły ocenić stworzony model oraz wnioskować, że z jego pomocą można zniwelować pewne klasy ataków na sieć obliczeniową.

    Na potrzeby pracy powstał prototyp systemu. Prototyp wykorzystuje lekkie maszyny wirtualne jako kontenery dla aplikacji prowadzących obliczenia. Wykorzystana została technologia Oracle VirtualBox. Protokół BitTorrent używany jest w celu zoptymalizowania transferu danych maszyny wirtualnej z serwera do węzłów. Całość prototypu zaimplementowana jest w technologii Node.js oraz wykorzystuje bazę danych CouchDB.

    Praca zakończona jest podsumowaniem. Zidentyfikowano problemy które wyszły poza obszar tej pracy oraz przedstawiono możliwe kierunku rozwoju sieci opartej o wolontariat.
}
